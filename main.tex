\documentclass[10pt]{article}
\usepackage[utf8]{inputenc}
 \setlength{\parindent}{10px}


\title{\textbf{Un algoritmo de dos fases para reconocer actividades humanas en el contexto de la industria 4.0 y procesos impulsados por el ser humano}}

\author{Borja Bordel$^1$, Ramón Alcarria$^1$, Diego Sánchez-de-Rivera$^1$ \\
\\
$^1$ Universidad Politécnica de Madrid,\\
Madrid, España\\
bbordel@dit.upm.es, ramon.alcarria@upm.es, diegosanchez@dit.upm.es}

\date{}

\begin{document}

\pagestyle{empty}

\maketitle

\thispagestyle{empty}

\subsection*{}
\textbf{Abstracto.} Futuros sistemas industriales, una revolución conocida como Industria 4.0, son concebidos para integrar personas en el mundo cibernético como prosumidores (proveedores de servicios y consumidores). En este contexto, los procesos impulsados por el ser humano aparecen como una realidad esencial y como herramientas para crear bucles de feedback entre el subsistema social (personas) y el subsistema cibernético (componentes tecnológicos) que son requeridos. Aunque se han propuesto diferentes herramientas, hoy en día las técnicas de reconocimiento de patrones son las mas prometedoras. Sin embargo, estas soluciones presentan algunos problemas pendientes importantes. Por ejemplo, dependen del hardware seleccionado para para adquirir información de e usuarios; o presentan un límite en la precisión del proceso de reconocimiento. Para abordar esta situación, en este paper está propuesto un algoritmo de dos fases para integrar personas en el sistema de la industria 4.0 y procesos impulsados por el ser humano. El algoritmo define acciones complejas como composiciones de movimientos simples. Las acciones complejas están reconocidas como Modelos Ocultos de Markov, y los movimientos simples están identificados usando Deformaciones Dinámicas en el Tiempo. De esa manera, solo los movimientos dependen de los dispositivos de hardware contratados para obtener información, y la precisión del reconocimiento de acciones complejas se ve enormemente incrementada. Una validación experimental real también es llevada a cabo para evaluar y comparar el rendimiento de la solución propuesta. \\
\\
\textbf{Palabras clave:} Industria 4.0, reconocimiento de patrones, Deformación Dinámica del Tiempo, Inteligencia artificial, Modelos ocultos de Markov

\section{Introducción}

Industria 4.0 [1] se refiere al uso de sistemas Ciber-Físicos (uniones de procesos físicos y cibernéticos) [2] como el principal componente digital en futuras soluciones digitales, principalmente (pero no solamente) en escenarios industriales. Generalmente, la digitalización ha causado, al final, el recambio de mecanismos de trabajo tradicionales por nuevos medios digitales. Por ejemplo, trabajadores en cadenas de montaje fueron sustituidos por robots durante la tercera revolución industrial.
\\
Sin embargo. Algunas aplicaciones industriales no pueden estar basadas en soluciones tecnológicas, el trabajo humano es todavía esencial [3]. Productos hechos a mano son un ejemplo de aplicaciones donde la presencia de trabajo manual es esencial. Estos sectores industriales, en cualquier caso, deben también ser integrados en la cuarta revolución industrial. De la unión de sistemas Ciber-Físicos (CPS) y humanos actuando como proveedores de servicio (trabajos activos), CPS humanizados surgen [4]. En estos nuevos sistemas, están permitidos los procesos impulsados por el ser humano [5]; e.g. procesos que son conocidos, ejecutados y organizados por personas (aunque quizás estén siendo vigilados por mecanismos digitales).
\\
Para crear una integración real entre personas y tecnología, y trasladar el proceso de ejecución del subsistema social (humanos) al mundo cibernético (componentes de hardware y software), son necesarias técnicas de extracción de información. Muchos diferentes enfoques y soluciones han sido citados durante los últimos años, pero hoy en día las técnicas de reconocimiento de patrones son las más prometedoras.
\\
El uso de inteligencia artificial, modelos estadísticos y otras herramientas similares han permitido increíble y real desarrollo de soluciones de reconocimiento de patrones, pero algunos retos siguen en el aire.
\\
En primer lugar, las habilidades de reconocimiento de patrones dependen del dispositivo de hardware subyacente para la captura de información. La estructura y procesos de aprendizaje cambian si (por ejemplo) en lugar de acelerómetros consideramos sensores de infrarrojo. Esto es muy problemático ya que la tecnología del hardware evoluciona mucho más rápido que las soluciones de software.
\\
Y, en segundo lugar, hay un límite para la precisión del proceso de reconocimiento. De hecho, mientras los actos humanos se vuelven más complicados, más variables y modelos complejos se requieren para reconocerlos. Este enfoque genera grandes problemas de optimización cuyo error residual es tan alto como el incremento del numero de variables; lo cual causa un declive en la tasa de reconocimiento [6]. En conclusión, las matemáticas (no el software, aunque depende de la implementación) fuerza una gran precisión para el proceso de reconocimiento de patrones dadas las acciones a estudiar. Para evitar esta situación, se debería considerar un menor número de variables, pero esto también reduce la complejidad de las acciones que puedan ser analizadas; una solución la cual no es aceptable en escenarios industriales donde las actividades de producción complejas son desarrolladas.
\\
Por lo tanto, el objetivo de este paper es describir un nuevo algoritmo de reconocimiento de patrones que guíe estos dos problemas básicos. El mecanismo propuesto define acciones como una composición de movimientos simples. Estos son reconocidos usando técnicas de deformaciones dinámicas en el tiempo (DTW) [7]. Este proceso depende del hardware seleccionado para la colecta de información; pero el DTW  es muy flexible y actualizando el repositorio del patrón es suficiente para configurar el algoritmo entero. Después, las acciones complejas son reconocidas como combinaciones de movimientos simples a través de Modelos Ocultos de Markov (HMM) [8]. Estos modelos son totalmente independientes de la tecnología hardware, ya que solo cuentan con acciones simples. Este enfoque de dos fases también reduce la complejidad de los modelos, incrementando la precisión y la tasa de éxito en el proceso de reconocimiento.
\\
El resto del paper se organiza de la siguiente manera: La sección 2 describe el estado del arte del reconocimiento de patrones para actividades humanas; La sección 3 describe la solución propuesta, incluyendo las dos fases definidas; La sección 4 presenta una validación experimental usando un escenario real y usuarios finales; y la sección 5 finaliza el paper.

\section{2 Estado del arte en el reconocimiento de patrones}



\section{Conclusiones y trabajos futuros}

En este paper presentamos un nuevo algoritmo de reconocimiento de patrones para integrar a personas en los sistemas de la  Industria 4.0 y en procesos impulsados por humanos. El algoritmo define actividades complejas como composiciones de movimientos. Las actividades complejas se reconocen usando los modelos de Cadenas de Markov y los movimientos simples usando Dynamic Time Warping. Para poder implementar este algoritmo en pequeños dispositivos integrados, seleccionamos ligeras configuraciones. También se lleva a cabo una validación experimental, y los resultados muestran una mejora global en la tasa de éxito en torno al 9%.
\\
Los futuros trabajos necesitaran una metodología más compleja para el procesado de datos, y la comparación de diferentes configuraciones del algoritmo propuesto serán evaluadas. Además, el propósito será analizado en diferentes escenarios.
\\
\textbf{Expresiones de gratitud.}La investicacion que conllevó a estos resultados ha recibido financiación del Ministerio de Economía y Competitividad a través del proyecto SEMOLA(TEC2015-68284- R) y del Ministerio de Ciencia, Innovación y Universidades a través del proyecto VACADENA (RTC-2017-6031-2).
\\
\textbf{Referencias} 
\begin{enumerate}
	\item	Bordel, B., Alcarria, R., Sánchez-de-  Rivera, D., & Robles, T. (2017, noviembre). Proteger los sistemas de la industria 4.0 contra los efectos maliciosos de los ataques ciberfísicos. En Conferencia Internacional sobre Computación Ubicua e Inteligencia Ambiental (págs. 161-171). Springer, Cham.
	\item Bordel, B., Alcarria, R., Robles, T., & Martín, D. (2017). Sistemas ciberfísicos: extensión de la detección generalizada de la teoría del control al Internet de las cosas. Informática omnipresente y móvil, 40, 156-184.
	\item	Neff, W. (2017). Trabajo y conducta humana. Routledge.
	\item	Bordel, B., Alcarria, R., Martín, D., Robles, T., & de Rivera, D. S. (2017). Autoconfiguración en sistemas ciberfísicos humanizados. Revista de Inteligencia Ambiental y Computación Humanizada, 8(4), 485-496.
	\item	Bordel, B., de Rivera, D. S., Sánchez-Picot, Á., & Robles, T. (2016). Control de procesos físicos en sistemas basados en la industria 4.0: un enfoque en los sistemas ciberfísicos. En Computación Ubicua e Inteligencia Ambiental (pp. 257-262). Springer, Cham.
	\item	Pal, S. K. y Wang, P. P. (2017). Algoritmos genéticos para el reconocimiento de patrones. Prensa CRC.
	\item	Muller, M. (2007). Deformación dinámica del tiempo. Recuperación de información para música y movimiento, 69-84.
	\item	Eddy, S. R. (1996). Modelos ocultos de markov. Opinión actual en biología estructural, 6(3), 361-365.
	\item	Kim, E., Helal, S. y Cook, D. (2010). Reconocimiento de la actividad humana y descubrimiento de patrones. IEEE Pervasive Computing/IEEE Computer Society [y] IEEE Communications Society, 9(1), 48.
	\item	Li, Z., Wei, Z., Yue, Y., Wang, H., Jia, W., Burke, L. E., ... y Sun, M. (2015). Un modelo de markov oculto adaptativo para el reconocimiento de actividad basado en un dispositivo multisensor portátil. Revista de sistemas médicos, 39(5), 57.
	\item	Ordóñez, F. J., Englebienne, G., De Toledo, P., Van Kasteren, T., Sanchis, A. y Krose, B. (2014). Reconocimiento de actividad en el hogar: inferencia bayesiana para modelos ocultos de Markov. Computación generalizada de IEEE, 13(3), 67-75.
	\item	Zhan, K., Faux, S. y Ramos, F. (2015). Campos aleatorios condicionales multiescala para reconocimiento de actividad en primera persona en ancianos y pacientes discapacitados. Informática móvil y generalizada, 16, 251-267.
	\item	Liu, A. A., Nie, W. Z., Su, Y. T., Ma, L., Hao, T. y Yang, Z. X. (2015). Campos aleatorios condicionales ocultos acoplados para reconocimiento de acción humana RGB-D. Procesamiento de señales, 112, 74-82.
	\item	iu, J., Huang, M. y Zhu, X. (julio de 2010). Reconocimiento de entidades biomédicas nombradas mediante campos aleatorios condicionales de salto de cadena. En Actas del taller de 2010 sobre procesamiento biomédico del lenguaje natural (págs. 10-18). Asociación de Lingüística Computacional.
	\item	Gu, T., Wu, Z., Tao, X., Pung, H. K. y Lu, J. (marzo de 2009). epsicar: un enfoque basado en patrones emergentes para el reconocimiento de actividades secuenciales, intercaladas y concurrentes. En Computación y comunicaciones generalizadas, 2009. PerCom 2009. Conferencia internacional IEEE sobre (págs. 1-9). IEEE.
	\item	Hu, B. G. (2014). ¿Cuáles son las diferencias entre los clasificadores bayesianos y los clasificadores de información mutua?. Trans. IEEE. Red neuronal Sistema de aprendizaje, 25(2), 249-264.
	\item	Wang, X., Liu, X., Pedrycz, W. y Zhang, L. (2015). Árboles de decisión basados en reglas difusas. Reconocimiento de patrones, 48(1), 50-59.
	\item	Davis, M. H. (2018). Modelos de Markov y    optimización. Routledge.
	\item Bordel Sánchez, B., Alcarria, R., Martín,   D., & Robles, T. (2015). TF4SM: un marco            para desarrollar soluciones de trazabilidad         en pequeñas empresas manufactureras.                Sensores, 15(11), 29478-29510.
\end{enumerate}



\end{document}
